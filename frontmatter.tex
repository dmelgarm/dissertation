%Frontmatter

\title{Seismogeodesy and Rapid Earthquake and Tsunami Source Assessment}

\author{Diego Melgar Moctezuma}
\degreeyear{2014}
\degreetitle{Doctor of Philosophy} 
\field{Earth Science}
\chair{Yehuda Bock}
%  The rest of the committee members  must be alphabetized by last name.
\othermembers{
T. Guy Masters\\ 
David T. Sandwell\\
Peter M. Shearer\\
Jose I. Restrepo\\
}
\numberofmembers{5} % |chair| + |cochair| + |othermembers|


%% START THE FRONTMATTER
%
\begin{frontmatter}

%% TITLE PAGES
%
%  This command generates the title, copyright, and signature pages.
%
\makefrontmatter 

%% DEDICATION
%
%  You have three choices here:
%    1. Use the ``dedication'' environment. 
%       Put in the text you want, and everything will be formated for 
%       you. You'll get a perfectly respectable dedication page.
%   
%
%    2. Use the ``mydedication'' environment.  If you don't like the
%       formatting of option 1, use this environment and format things
%       however you wish.
%
%    3. If you don't want a dedication, it's not required.
%
%
\begin{dedication} 
  To two, the loneliest number since the number one.
\end{dedication}


% \begin{mydedication} % You are responsible for formatting here.
%   \vspace{1in}
%   \begin{flushleft}
% 	To me.
%   \end{flushleft}
%   
%   \vspace{2in}
%   \begin{center}
% 	And you.
%   \end{center}
% 
%   \vspace{2in}
%   \begin{flushright}
% 	Which equals us.
%   \end{flushright}
% \end{mydedication}



%% EPIGRAPH
%
%  The same choices that applied to the dedication apply here.
%
\begin{epigraph} % The style file will position the text for you.
  \emph{A careful quotation\\
  conveys brilliance.}\\
  ---Smarty Pants
\end{epigraph}

% \begin{myepigraph} % You position the text yourself.
%   \vfil
%   \begin{center}
%     {\bf Think! It ain't illegal yet.}
% 
% 	\emph{---George Clinton}
%   \end{center}
% \end{myepigraph}


%% SETUP THE TABLE OF CONTENTS
%
\tableofcontents
\listoffigures  % Uncomment if you have any figures
\listoftables   % Uncomment if you have any tables



%% ACKNOWLEDGEMENTS
%
%  While technically optional, you probably have someone to thank.
%  Also, a paragraph acknowledging all coauthors and publishers (if
%  you have any) is required in the acknowledgements page and as the
%  last paragraph of text at the end of each respective chapter. See
%  the OGS Formatting Manual for more information.
%
\begin{acknowledgements} 
 Thanks to whoever deserves credit for Blacks Beach, Porters Pub, and
 every coffee shop in San Diego. 

Portions of this dissertation have been published in peer reviewed journals. Except for the discussion of accelerometer biases in Kalman filtering, Chapter 2 is published in its entirety in:
\begin{itemize}
\item \textbf{Melgar, D.}, Bock, Y., Sanchez, D., and Crowell, B.W., ``On Robust and Reliable Automated Baseline Corrections for Strong Motion Seismology'', \emph{J. Geophys. Res.}, 118(3), 2013.
\item Bock, Y., \textbf{Melgar, D.}, and Crowell, B.W., ``Real-Time Strong-Motion Broadband Displacements from Collocated GPS and Accelerometers'', \emph{Bull. Seism. Soc. Am.}, 101(6), 2011.
\end{itemize}
Chapter 3 has been published in its entirety in:
\begin{itemize}
\item \textbf{Melgar, D.}, Crowell, B.W., Bock, Y., and Haase, J.S, ``Rapid modeling of the 2011 Mw 9.0 Tohoku-oki earthquake with seismogeodesy'', \emph{Geophys. Res. Lett}, 40(12), 2013.
\item \textbf{Melgar, D.}, Bock, Y., and Crowell, B.W., ``Real-Time Centroid Moment Tensor Determination for Large Earthquakes from Local and Regional Displacement Records'', \emph{Geophys. J. Int.}, 188(2), 2012.
\end{itemize}
Chapter 4 consists of entirely unpublished material. Sections 5.1 through 5.4 of Chapter 5 have been published in
\begin{itemize}
\item \textbf{Melgar, D.} and Bock, Y. ``Near-Field Tsunami Models with Rapid Earthquake Source Inversions from Land- and Ocean-Based Observations: The Potential for Forecast and Warning'', \emph{J. Geophys. Res.}, 118(11), 2013.
\end{itemize}
and relies heavily on code modified from its original form as it appeared for the publication
\begin{itemize}
\item Crowell, B.W., Bock, Y., and \textbf{Melgar, D.}, ``Real-time inversion of GPS data for finite fault modeling and rapid hazard assessment'', \emph{Geophys. Res. Lett}, 39(9), 2012.
\end{itemize}
the remainder is unpublished material.


\end{acknowledgements}


%% VITA
%
%  A brief vita is required in a doctoral thesis. See the OGS
%  Formatting Manual for more information.
%
\begin{vitapage}
\begin{vita}
  \item[2009] B.~Eng. in Geophysics, Universidad Nacional Autonoma de Mexico
  \item[2010] M.~Sc in Earth Science University of California, San Diego
  \item[2014] Ph.~D. in Earth Science, University of California, San Diego 
\end{vita}
\begin{publications}
\item Crowell, B.W., \textbf{Melgar, D.}, Bock, Y., Haase, J.S, and Geng, J., ``Earthquake Magnitude Scaling using Seismogeodetic Data'', \emph{Geophys. Res. Lett}, 40(23), 2013.
\item \textbf{Melgar, D.} and Bock, Y. ``Near-Field Tsunami Models with Rapid Earthquake Source Inversions from Land- and Ocean-Based Observations: The Potential for Forecast and Warning'', \emph{J. Geophys. Res.}, 118(11), 2013.
\item Geng, J., \textbf{Melgar, D.}, Bock, Y., Pantoli, E., and Restrepo, J.I., ``Recovering coseismic point ground tilts from collocated high-rate GPS and accelerometers'', \emph{Geophys. Res. Lett}, 40(19), 2013.
\item \textbf{Melgar, D.}, Pantoli, E., Bock, Y., and Restrepo, J.I., ``Displacement Acquisition for the NEESR:BNCS Building Shaketable Test via GPS Sensors'', \emph{Network for Earthquake Engineering Simulation (distributor)}, DOI:10.4231/D3V97ZR5H, 2013.
\item \textbf{Melgar, D.}, Crowell, B.W., Bock, Y., and Haase, J.S, ``Rapid modeling of the 2011 Mw 9.0 Tohoku-oki earthquake with seismogeodesy'', \emph{Geophys. Res. Lett}, 40(12), 2013.
\item Geng, J., Bock, Y., \textbf{Melgar, D.}, Crowell, B.W., and Haase, J.S, ``A new seismogeodetic approach applied to GPS and accelerometer observations of the 2012 Brawley seismic swarm: Implications for earthquake early warning'', \emph{Geochem. Geophys. Geosyst}, 14(7), 2013.
\item \textbf{Melgar, D.}, Bock, Y., Sanchez, D., and Crowell, B.W., ``On Robust and Reliable Automated Baseline Corrections for Strong Motion Seismology'', \emph{J. Geophys. Res.}, 118(3), 2013.
\item Perez-Campos, X., \textbf{Melgar, D.}, Singh, S.K., Cruz-Atienza, V., Iglesias, A., and Hjorleifsdottir, V., ``Determination of tsunamigenic potential of a scenario earthquake in the Guerrero seismic gap along the Mexican subduction zone'', \emph{Seism. Res. Lett}, 84(3), 2013.
\item Crowell, B.W., Bock, Y., and \textbf{Melgar, D.}, ``Real-time inversion of GPS data for finite fault modeling and rapid hazard assessment'', \emph{Geophys. Res. Lett}, 39(9), 2012.
\item Singh, S.K., Perez-Campos, X., Iglesias, A., \textbf{Melgar, D.}, ``A Method for Rapid Estimation of Moment Magnitude for Early Tsunami Warning Based on Coastal GPS Networks'', \emph{Seism. Res. Lett}, 83(3), 2012.
\item \textbf{Melgar, D.}, Bock, Y., and Crowell, B.W., ``Real-Time Centroid Moment Tensor Determination for Large Earthquakes from Local and Regional Displacement Records'', \emph{Geophys. J. Int.}, 188(2), 2012.
\item Bock, Y., \textbf{Melgar, D.}, and Crowell, B.W., ``Real-Time Strong-Motion Broadband Displacements from Collocated GPS and Accelerometers'', \emph{Bull. Seism. Soc. Am.}, 101(6), 2011.
\item \textbf{Melgar, D.} and Perez-Campos, X., ``Imaging the Moho and Subducted Oceanic Crust at the Isthmus of Tehuantepec, Mexico, from Receiver Functions'', \emph{Pure Appl. Geophysics}, 168, 2010.

\end{publications}
\end{vitapage}


%% ABSTRACT
%
%  Doctoral dissertation abstracts should not exceed 350 words. 
%   The abstract may continue to a second page if necessary.
%
\begin{abstract}
  This dissertation will be abstract. 
\end{abstract}


\end{frontmatter}
