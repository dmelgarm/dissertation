%Chapter 3: Rapid Static Modeling

\chapter{Rapid Static Modeling}

In the introductory chapter we discussed how recent response to large earthquakes has been hampered by a strict reliance on seismic data of early characterization of the seismic source. In Chapter 2 we discussed briefly the problems facing traditional seismological instrumentation at regional distances of large events. Seismometers clip during strong shaking and strong motion sensors are beleaguered by baseline offsets. The latter problem, baseline offsets, is particularly important because it affects the long period band of a seismic time series. It's precisely this frequency band that is most useful for discerning the broad features of large earthquakes. If a geophysical instrument is band limited then it will become increasingly difficult to differentiate between say a magnitude 7 and a magnitude 8 event. This is a condition known as saturation and it's common in early warning algorithms that rely on seismometers and accelerometers \cite{Brown2011}.

Large events induce a permanent deformation of the Earth's crust, the static field. If the earthquake is large enough, and the geophysical sensor close enough or sensitive enough it is possible to measure the static field. For hazards applications this is of great interest because the static field is a zero frequency wave and as such the longest period information we can measure about an event. Thus characterization of the static field and of its contribution to a seismogram solves the problem of saturation. Furthermore the static field can be modeled to have no time dependence, this makes it simple to extend earthquake source models from point sources to more realistic geometries

Although noisier than accelerometers, GPS can easily record the permanent deformation at regional distances of large earthquakes, this was documented as far back as the 1992 M7.2 Landers earthquakend \citep{bock1993,blewitt1993}. Throughout this chapter we will show how such recordings can be used for rapid estimation of centroid moment tensors (CMTs), line source moment tensors and slip inversions.

The Kalman filter method discussed in the previous Chapter aides these algorithms insofar as it allows for better quantification of the vertical static offset. However, its effect, while non-negligble, is not of paramount importance. It's not until later on in Chapter 4 where we will see the true benefit of the filtered data for source estimation when we compute kinematic models. However, when pertinent, we will demonstrate the improvements to the results as a consequence of the filtering operation.

\section{Centroid Moment Tensor Inversion}
\label{sec:cmt}

Computation of the seismic moment tensor (MT) for a given earthquake is one of the fundamental kinds of modeling that can be performed. The moment tensor can be calculated from a number of methods such as polarity of first arrivals \citep{Havskov2010} or waveform matching \citep{dreger2003} and is a compact representation of the earthquake source that contains basic information on the size of the event, the fault plane geometry and the style of faulting. Moment tensor solutions are of use over a range of earthquake magnitudes. Small to medium events are utilized for tectonic studies and to determine the stress regime within a region. For large events, rapid determination of the centroid location as well as the moment tensor (CMT) provides valuable information for earthquake response, tsunami early warning and as a starting point for finite fault source modeling.

Currently there are a number of efforts that routinely compute moment tensor solutions for earthquakes worldwide. The most comprehensive catalogue of such solutions is contained in the Global Centroid Moment Tensor (GCMT) Project. At its inception the GCMT project included inversion of body and surface waves \citep{dziewonski1981,dziewonski1983} and has seen numerous refinements since such as inclusion of aspherical Earth structure, attenuation, etc. This method however employs only teleseismic data and its emphasis is in data collection and catalogue compilation not in rapid modeling. Real-time moment tensors can be obtained for small to medium events using time domain waveform matching inversion schemes  \citep{dreger1990,dreger1993}. However, real-time CMT determination of medium to large events is still an active area of research.

One of the most important advances in computing CMTs as quickly as possible for large events is contained in the work of \citet{kanamori2008} who elaborated on \citet{kanamori1993}'s observation of the W phase, a long period phase arriving in between the direct P and S waves. They showed that inversion for the moment tensor using data as close as 15 from the source is viable. W phase inversion algorithms currently run in real time at the USGS, Pacific Tsunami Warning Center (PTWC) and Institut du Physique du Globe de Strasbourg (IPGP-EOST) \citep{hayes2009}. Since the W phase arrives well before large amplitude surface waves and remains on-scale far longer such inversion algorithms have shown to be a marked improvement in rapid computation of moment tensor solutions for large events over traditional waveform matching techniques. 

Following the Mw 9.0 Tohoku-oki earthquake,\citet{duputel2011} showed that it was feasible to use data at distances as small as 11 from the source. However, W phase based inversion schemes, while very robust, require long period displacement records (e.g. 200-1000s for the 2011 Tohoku-oki event) \citep{duputel2011}. Furthermore, as discussed before, they are almost always unusable close to the source in real time for well-known reasons; velocity instruments clip and it is difficult to extract long period motions from strong-motion accelerometer data in real time because of tilts and rotations of the instruments.

Thus, there seems to be a limitation in how fast moment tensor solutions can be obtained operationally for large events using seismic instruments and existing seismological methods. For example, for the Tohoku-oki event it took 20 minutes after origin time to arrive at the first CMT solutions by agencies running W phase algorithms \citep{duputel2011}, even though the rupture had a duration of three minutes \citep{simons2011}. This delay was due to the reliance on teleseismic data. After several iterations using progressively more data, the final CMT solution \citep{hayes2011} was obtained 90 minutes after origin time using data up to 90 from the rupture. The first estimate of moment magnitude was obtained in about 3 minutes by the Japan Meteorological Agency (JMA), but was grossly underestimated at Mw=6.8. Duputel et al. (2011a) documented that in the numerous iterations between agencies the nodal planes were somewhat consistent with only minor variations in strike dip and rake, the magnitudes oscillated between Mw 8.8-9.0 after the 20-minute mark, but the centroid locations varied by as much as 2 and 60 km in depth.

%
There have been some attempts to address the use of GPS static deformation estimates into rapid source modeling, notably by Blewitt et al. (2006) who showed that given an epicentral location and assuming thrust faulting for the 2004 Mw 9.2  Sumatra-Andaman earthquake, one could have estimated an accurate magnitude within 15 minutes of the origin time using global GPS stations at regional to teleseismic distances. 
We present a robust method for determining CMT solutions that is considerably faster than current seismic methods, based on real-time high-rate displacement data from near-source GPS stations. Although we are not explicitly solving for the style and geometry of faulting, that information is implicit in the moment tensor solution. In general, we do not require prior knowledge of the sense or extent of faulting, although that information could be used if available.
We demonstrate the new algorithm by replaying the estimation of 1 Hz displacements for the 2003 Mw 8.3 Tokachi-oki earthquake using GPS data from Japan�s GPS network (GEONET) (Miyazaki et al., 1998) and for the 2010 Mw 7.2 El Mayor-Cucapah earthquake using data from the California Real Time Network (CRTN) (Bock et al., 2011). GPS data for the 2011 Mw 9.0 Tohoku-oki earthquake at the time of this study were not yet in the public domain, but our algorithm can be easily extended to this great event.


%\appendix
%\chapter{Final notes}
%  Remove me in case of abdominal pain.

